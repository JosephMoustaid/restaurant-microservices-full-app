\chapter*{Résumé}
\thispagestyle{empty}

\section*{Résumé}

Ce projet présente \textbf{Gourmet Gateway}, une application web complète de gestion de restaurants développée selon une architecture microservices moderne. Le système permet aux utilisateurs de découvrir des restaurants, effectuer des réservations et rechercher des établissements à proximité grâce à l'intégration de l'API Google Places.

L'application démontre les meilleures pratiques en matière de systèmes distribués, incluant la découverte de services (Eureka), le routage via une passerelle API (Spring Cloud Gateway), l'authentification basée sur JWT, et une interface web réactive développée avec React 19 et TypeScript.

Le backend est composé de six microservices indépendants développés avec Spring Boot : un serveur Eureka pour la découverte de services, une passerelle API, un service de gestion des restaurants, un service de réservations, un service de recherche de lieux et un service d'authentification des utilisateurs. Chaque service possède sa propre base de données PostgreSQL, respectant le principe "database per service".

Le frontend, développé avec React 19, TypeScript et Vite, offre une interface utilisateur moderne et responsive avec des fonctionnalités de gestion CRUD complètes, des tableaux de bord analytiques et une intégration cartographique.

\keywords{Microservices, Spring Boot, Spring Cloud, React, TypeScript, API Gateway, Eureka, JWT, PostgreSQL, Architecture distribuée}

\section*{Abstract}

This project presents \textbf{Gourmet Gateway}, a comprehensive restaurant management web application built using modern microservices architecture. The system enables users to discover restaurants, make reservations, and search for nearby dining locations through Google Places API integration.

The application demonstrates industry best practices in distributed systems design, featuring service discovery (Eureka), API gateway routing (Spring Cloud Gateway), JWT-based authentication, and a reactive web frontend developed with React 19 and TypeScript.

The backend consists of six independent microservices built with Spring Boot: a Eureka server for service discovery, an API gateway, a restaurant management service, a reservations service, a places search service, and a user authentication service. Each service maintains its own PostgreSQL database, following the "database per service" principle.

The frontend, developed with React 19, TypeScript, and Vite, provides a modern and responsive user interface with complete CRUD functionality, analytical dashboards, and map integration.

\keywordss{Microservices, Spring Boot, Spring Cloud, React, TypeScript, API Gateway, Eureka, JWT, PostgreSQL, Distributed Architecture}
