Ce projet a permis de concevoir et développer \textbf{Gourmet Gateway}, une application web de gestion de restaurants basée sur une architecture microservices. L'objectif était de démontrer la mise en œuvre des principes d'architecture distribuée moderne.

\section*{Réalisations}

Architecture microservices opérationnelle avec six services indépendants (Eureka, Gateway, Restaurant, Reservation, Places, User), chacun avec sa base PostgreSQL. Intégration réussie de technologies modernes : Spring Boot/Cloud, React 19/TypeScript, JWT/BCrypt, Google Places API. Interface utilisateur responsive avec tableaux de bord analytiques, gestion CRUD complète et recherche géolocalisée.

\section*{Compétences acquises}

Conception d'architecture microservices, patterns de communication inter-services, gestion de découverte de services, Spring Boot/Cloud/WebFlux/Security, React avec TypeScript, sécurité JWT/BCrypt, intégration API externes.

\section*{Défis et solutions}

\textbf{Complexité distribuée} : Ordre de démarrage géré par scripts, configuration Eureka/Gateway rigoureuse.

\textbf{Cohérence des données} : Validation inter-services via OpenFeign, acceptation de cohérence éventuelle.

\textbf{Sécurité CORS} : Configuration appropriée à la Gateway, gestion JWT en localStorage.

\section*{Perspectives}

Production : Circuit breakers, logs centralisés, tracing distribué, monitoring.

Scalabilité : Docker, Kubernetes, cache Redis, message queues.

Fonctionnalités : Système de notation, recommandations ML, notifications, paiement, application mobile.

\section*{Conclusion}

Le projet Gourmet Gateway démontre qu'une architecture microservices bien conçue répond aux défis de scalabilité, maintenance et évolution. Les objectifs sont atteints : architecture distribuée fonctionnelle, intégration technologique réussie, interface utilisateur complète. Les compétences acquises sont directement applicables en contexte professionnel.
